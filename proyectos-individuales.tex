\documentclass[
    type=individual,
    yellowbox
]{memoriact}

\begin{document}

\maketitle

\section{Datos del proyecto}

\textbf{IP 1} (Nombre y apellidos): \\
\textbf{IP 2} (Nombre y apellidos): 

\textbf{TÍTULO DEL PROYECTO (ACRÓNIMO)}: \\
\textit{\textbf{TITLE OF THE PROJECT (ACRONYM)}}: 

\section{Justificación y novedad de la propuesta}

\subsection{Adecuación de la propuesta a las características y finalidad de la modalidad seleccionada}

Se deberá justificar la adecuación de la propuesta a las características de la convocatoria, así como la adecuación de la investigación a desarrollar a la modalidad de proyecto seleccionada. Recuerde que:

\begin{itemize}[label=--]
    \item Los proyectos de la modalidad de investigación no orientada son proyectos sin orientación temática previamente definida, que tienen como objetivo primordial avanzar en el conocimiento, independientemente del horizonte temporal y su ámbito de aplicación.
    \item Los proyectos de la modalidad de investigación orientada son proyectos que están orientados a la resolución de problemas concretos y vinculados a las prioridades temáticas asociadas a los desafíos mundiales, incluidas en el Plan Estatal de Investigación Científica. Técnica y de Innovación 2024-2027 y en el Anexo III de la convocatoria.
\end{itemize}

\subsection{Justificación y contribución esperada del proyecto a la generación de conocimiento en la temática de la propuesta. Hipótesis de partida}

Se deberá explicar la motivación de la propuesta del proyecto en el contexto de los conocimientos científico-técnicos de la materia específica o línea de investigación del mismo, debiendo quedar clara la novedad de la contribución esperada del trabajo que se propone para el ámbito temático en el que se enmarca.

Se deberá indicar la hipótesis de partida y su novedad en relación con el estado del arte de la temática de la propuesta.

Si el proyecto es continuación de otro previamente financiado, deben indicarse con claridad los objetivos y los resultados ya alcanzados, de manera que sea posible evaluar el avance real que se propone en el nuevo proyecto.

\textbf{\textit{\underline{En los proyectos de modalidad de investigación orientada}:}}

\subsection{Justificación y contribución esperada del proyecto a solucionar problemas concretos vinculados a la prioridad temática seleccionada}

Si la memoria se presenta en la modalidad de investigación orientada, se deberá identificar la prioridad temática, de las recogidas en el anexo III, y justificar la selección de la misma.

Se deberá identificar claramente la \textbf{contribución} de la propuesta a solventar los problemas o necesidades de la prioridad temática seleccionada.

Si dentro de la \textbf{modalidad de investigación orientada se presenta al tipo RTA}, se deberá, además, indicarla línea de investigación prioritaria, de las recogidas en el anexo VI y justificar la contribución de la propuesta a la misma.

\section{Objetivos, metodología y plan de trabajo}

\subsection{Objetivos generales y específicos}

Se deberán describir los \textbf{objetivos generales y específicos}, enumerándolos brevemente, con claridad, precisión y de manera realista (acorde con la duración prevista del proyecto y la composición del equipo de investigación y/o de trabajo). En los proyectos con dos  investigadores/as principales, deberá indicarse expresamente de qué objetivos específicos se hará responsable cada uno/a de ellos/as.

\subsection{Descripción de la metodología}

Describa con detalle la \textbf{metodología} propuesta en relación con los objetivos y con el estado del arte, resaltando la novedad y/o adecuación de la misma y destacando aquellas \textbf{etapas críticas} cuyo resultado pueda afectar a la viabilidad del plan de trabajo previsto o requerir un reajuste del mismo.

\subsection{Plan de trabajo y cronograma}

Se deberá incluir un \textbf{cronograma} claro y preciso de las fases e hitos previstos en relación con los objetivos planteados en la propuesta, con indicación de la responsabilidad y la participación de cada uno de los miembros de equipo investigador y del equipo de trabajo. Se deberá indicar la especialización temática de los miembros de ambos equipos, así como los miembros claves para la ejecución de cada una de las tareas previstas. Se recuerda que los miembros del equipo de trabajo no podrán figurar como responsables de objetivos y tareas del proyecto.

\subsection{Identificación de puntos críticos y plan de contingencia}

Se deberán indicar aquellas \textbf{etapas críticas} de la ejecución de la propuesta cuyo resultado pueda afectar a la viabilidad del plan de trabajo previsto o requerir un reajuste del mismo.

Se deberá incluir una evaluación crítica de las posibles dificultades para alcanzar alguno de los objetivos específicos y un \textbf{plan de contingencia} para resolverlas.

\subsection{Resultados previos del equipo en la temática de la propuesta}

Describa brevemente los antecedentes y contribuciones previas del equipo de investigación en la temática de la propuesta que justifiquen su capacidad para llevarla a cabo y avalen la viabilidad de la misma.

\subsection{Recursos humanos, materiales y de equipamiento disponibles para la ejecución del proyecto} 

Describa los medios \textbf{materiales, infraestructuras y/o equipamientos singulares} a disposición del proyecto que permitan abordar la metodología propuesta.

\section{Impacto esperado de los resultados}

En la aplicación informática de solicitud deberá introducir una descripción, de un máximo de  3.500 caracteres, sobre el impacto esperado de los resultados del proyecto, cuyo contenido podrá ser publicado a efectos de difusión si el proyecto fuera financiado en esta convocatoria.

\subsection{Impacto esperado en la generación de conocimiento científico-técnico en el ámbito temático de la propuesta}

Se deberá incluir una descripción del \textbf{impacto científico-técnico} que se espera de los
resultados del proyecto, tanto a nivel nacional como internacional, identificando, en su caso, aquellos resultados que permitan avanzar en el conocimiento científico-técnico de carácter interdisciplinar.

En el caso de que el proyecto se presente en la \textbf{modalidad investigación orientada} se deberá identificar la contribución de los resultados esperables a la prioridad temática seleccionada.

\subsection{Impacto social y económico de los resultados previstos}

Se deberá describir el impacto social y económico y los beneficios que puedan derivarse
de los resultados de la propuesta y de sus aplicaciones en términos de generar valor tangible e intangible para la sociedad, la economía, la cultura o las políticas públicas, incluido el impacto sobre la creación de empleo (más allá de la contratación de personal para la realización del propio proyecto).

Incluya una descripción del impacto de los resultados en la dimensión de género o el impacto asociado al ámbito de la discapacidad y otras áreas de inclusión social, y cualquier otro aspecto de permita valorar el beneficio de las actividades propuestas para la sociedad.

Algunas de las preguntas que se podrán contestar en este punto son, por ejemplo, ¿en qué medida los resultados previstos en el proyecto contribuirán al bienestar de la población?,¿en qué medida permitirán fortalecer la competitividad, el desarrollo tecnológico y la productividad de las empresas?,¿en qué medida contribuirán a la transferencia del conocimiento?, ¿en qué medida permitirán aumentar la seguridad de la sociedad (seguridad alimentaria, eventos medioambientales extremos, ciber seguridad, etc.)?, ¿en qué medida contribuirán a la mejora o a la protección del medioambiente y/o a los objetivos de desarrollo sostenible definidos por las Naciones Unidas?, ¿en qué medida contribuirán a un mejor conocimiento de su entorno y a la consecución de los retos sociales definidos por la UE y/o de los objetivos de la Estrategia Española de Ciencia, Tecnología e Innovación? ¿en qué medida contribuirán a la generación de empleo, capacitación, etc.?, ¿en qué medida permitirán mejorar la formación de la población?

\subsection{Plan de comunicación científica e internacionalización de los resultados (indicar la previsión de publicaciones en acceso abierto)}

Deberá incluir un \textbf{plan de comunicación científica} de los resultados del proyecto, poniendo énfasis en aquellas actuaciones de especial relevancia y diferentes a las habituales.

Se deberá incluir un plan de internacionalización de los resultados, en el que se podrá incluir entre otras actuaciones, futuras colaboraciones con grupos de I+D+I internacionales académicos o industriales, participaciones en proyectos y contratos internacionales, etc. que puedan ser consecuencia de los resultados del proyecto.

Se deberá indicar la previsión sobre publicaciones en acceso en abierto.

\subsection{Plan de divulgación de los resultados a los colectivos más relevantes para la temática del proyecto y a la sociedad en general}

Se deberá incluir un plan de divulgación de los resultados del proyecto a los colectivos más relevantes para la temática del proyecto y a la sociedad en general.

\textbf{\textit{\underline{En los casos en que sea de aplicación}:}}

\subsection{Plan de transferencia y valorización de los resultados}

Si considera que pueden alcanzarse conocimientos y/o resultados susceptibles de \textbf{transferencia}, se deberán identificar los conocimientos/resultados potencialmente transferibles y detallar el plan previsto para la transferencia de los mismos, así como indicar las posibles entidades interesadas en ellos (especialmente si el proyecto es de la modalidad de investigación orientada).

La documentación relativa a las entidades interesadas en los resultados del proyecto no debe presentarse con la solicitud, sino que debe quedar en poder de la entidad solicitante tal como establece la convocatoria, debiendo aportar dicha documentación solo en caso de serle requerida.

\subsection{Resumen del plan de gestión de datos previsto}

Deberá realizar una previsión del plan de gestión de datos de investigación en el que se indique qué datos se van a recoger o generar (tipologías y formatos), cómo será el acceso (quién, cómo y cuándo se podrá acceder a los datos) y en qué repositorio está previsto que se depositen. En el caso de datos que estén sometidos a la reglamentación de protección de los datos personales o de aspectos éticos, indique cómo se gestionarán. En el caso de los proyectos que resulten financiados, durante la ejecución del proyecto y junto al informe final se podrá solicitar un plan de gestión de datos formal completo.

\subsection{Efectos de la inclusión de género en el contenido de la propuesta}

Recuerde que, si en el contenido de la investigación propuesta se contemplan aspectos que pudieran tener una dimensión de sexo o género, por su temática, metodología, resultados o aplicaciones, estos deberán desarrollarse en los apartados correspondientes de la memoria científico-técnica.

Deberá describir cómo se integra el análisis de sexo y/o género en todas las fases del ciclo de la investigación: hipótesis, metodología, ética.

Deberá describir el impacto, si lo hay, de los resultados del proyecto en la dimensión de sexo/género, Así mismo, se deberá desarrollar cómo los resultados o aplicaciones del proyecto puedan verse afectados (in)directamente en función del sexo y/o género.

\section{Justificación del presupuesto solicitado}

Describa una justificación cualitativa de la necesidad del presupuesto solicitado para la
ejecución del proyecto de I+D+I.

Si se solicita ayuda para la contratación de personal, justifique su necesidad, la titulación y/o formación requerida e incluya la descripción de las tareas que vaya a realizar.

En caso de solicitar algún equipamiento esencial para la ejecución de las tareas propuestas, justifique su necesidad y, en su caso, la novedad técnica y/o metodológica que aportaría.

\section{Capacidad formativa}

Este apartado sólo se rellenará si se solicita una o varias de las ayudas a actuaciones para la formación de personal investigador predoctoral asociadas a los proyectos. La concesión de estas ayudas sólo será posible en un número limitado de los proyectos aprobados.

Esta información se tendrá en cuenta para valorar la capacidad formativa del equipo, con independencia de la valoración científica que reciba el proyecto.

\subsection{Programa de formación previsto en el contexto del proyecto solicitado}

Incluya un breve resumen del programa de formación previsto en el contexto del proyecto solicitado, con indicación del programa de doctorado concreto y la universidad dónde se cursará, cursos de especialización y estancias breves previstas.

\subsection{Tesis realizadas o en curso en el ámbito del equipo de investigación (últimos 10 años)}

Incorpore una relación de tesis realizadas o en curso en el ámbito del equipo de investigación (en los últimos 10 años) con indicación del nombre del/la doctorando/a, el título de tesis, la fecha de inicio y de obtención del grado de doctor o de la fecha prevista de lectura de tesis y las publicaciones en las que figura el contratado predoctoral.

\subsection{Desarrollo científico o profesional de los doctores/as egresados/as}

Incorpore una breve descripción del desarrollo científico o profesional de los doctores egresados del equipo de investigación durante los últimos 10 años.


\section{Condiciones específicas para la efecución de determinados proyectos}

Este apartado sólo se rellenará si en la aplicación electrónica de solicitud se contesta afirmativamente a alguno de los aspectos relacionados con las condiciones o implicaciones recogidas en el Anexo IV, tales como ética, bioseguridad, experimentación animal, ensayos clínicos, utilización de células o tejidos humanos, células troncales embrionarias,  ealización de encuestas cuantitativas en el ámbito de las ciencias sociales, actividades arqueológicas, recursos genéticos españoles o extranjeros, etc., y únicamente en aquellos supuestos que no se contemplen en la aplicación de solicitud y que afecten a las actividades previstas en la propuesta presentada.

En la aplicación informática de \textbf{solicitud} deberá incluir, en caso de contestar “Sí” a alguna de estas implicaciones, una justificación de un máximo de 1.000 caracteres y las autorizaciones necesarias de las que dispone para su ejecución, con una justificación de 500 caracteres como máximo. En todo caso, si considera necesario ampliar dicha información, puede hacerlo en este apartado de la memoria técnica.

La información que se recomienda aportar se refiere a:

\begin{enumerate}[label=\alph*)]
    \item Una descripción de los aspectos específicos referidos a la investigación que se propone.
    \item Una explicación de las consideraciones, procedimientos o protocolos que prevé aplicar en el proyecto en cumplimiento de la normativa vigente, en cada caso.
    \item Una indicación de las instalaciones que son preceptivas y de las que dispondrá en la institución donde se ejecutará el proyecto. Y en su caso, su previsión de uso acceso para aquellas instalaciones de las que no dispone.
    \item La relación de autorizaciones preceptivas de las que ya dispone, o que se encuentren en tramitación, para las actividades previstas en el proyecto
\end{enumerate}

Las autorizaciones y demás documentación relativa a estos extremos no deben presentarse con la solicitud, sino que deben quedar en poder de la entidad solicitante (tal como establece la convocatoria), debiendo aportar dicha documentación sólo en caso de serle requerida durante la ejecución del proyecto o con el informe intermedio y/o final de seguimiento científico técnico.

\bibliographystyle{plain}  
\bibliography{references}

\end{document}
